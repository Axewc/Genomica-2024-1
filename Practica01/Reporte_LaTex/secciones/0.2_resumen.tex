\small

\vspace{11pt}

\centerline{\rule{0.95\textwidth}{0.4pt}}

\begin{center}
    
    \begin{minipage}{1\textwidth}
        En el estudio de muestras de ventilas hidrotermales, se analizaron fragmentos de DNA para identificar y entender la presencia de genes y posibles organismos que habitan estos ambientes extremos. Utilizando herramientas bioinformáticas, se leyeron secuencias en formato FASTA y se identificaron marcos de lectura abiertos (ORFs). Los ORFs se tradujeron en secuencias de aminoácidos para inferir funciones potenciales de las proteínas. Se determinó que uno de los fragmentos de DNA tenía una proteína relacionada con la formación del núcleo celular, sugiriendo la presencia de organismos eucariotas en la muestra. Esta revelación desafía el entendimiento tradicional de los tipos de vida que pueden soportar tales ambientes extremos. Adicionalmente, se discutió la complejidad de recrear organismos a partir de secuencias de DNA, destacando los desafíos técnicos y éticos, y el porqué aún no tenemos un "Jurassic Park"  en la realidad.
    
        \vspace{4mm}
        % PALABRAS CLAVE
        %\noindent \textbf{Palabras clave:} Palabra clave 1, palabra clave 2, palabra clave 3, palabra clave 4.
    
    \end{minipage}
    
\end{center}

\centerline{\rule{0.95\textwidth}{0.4pt}}

\vspace{15pt}