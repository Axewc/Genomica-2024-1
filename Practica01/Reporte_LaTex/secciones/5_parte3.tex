%%%%%%%%%%%%%%%%%%%%%%%%%%%%%%%%%%%%%%%%%%%%%%%%
\section*{Ensayo (Parte3)} % NOMBRE DE LA SECCIÓN
\label{sec:parte3} % ETIQUETA
%%%%%%%%%%%%%%%%%%%%%%%%%%%%%%%%%%%%%%%%%%%%%%%%

\textbf{Pregunta genuina:} Si el DNA contiene toda la información que nos conforma como organismos vivos, como especie y como organismos únicos, ¿Por qué no existe \textit{Jurassic Park/World}? ¿Es fácil crear un organismo en el laboratorio solo con tener una cadena DNA?


\subsection*{¿Por qué aún no tenemos un Jurassic Park?}

¡Hola! Si eres de los que crecieron viendo "Jurassic Park" y soñando con ver un Tiranosaurios rex en vivo y en directo, te comprendo totalmente. Pero, aunque la idea de recrear dinosaurios y pasear entre ellos suena alucinante, la ciencia real detrás de esto es un poco más complicada que insertar un poco de ADN antiguo en un huevo y esperar a que salga un velociraptor. Así que, antes de que empieces a ahorrar para tu entrada al parque de los dinosaurios, aclaremos un par de cosas.

Primero, el ADN. Sí, es cierto que el ADN es como el manual de instrucciones que nos hace ser lo que somos. Piénsalo como ese complicado manual con el que armas un mueble. Pero, en lugar de tornillos y tablones, el ADN tiene las instrucciones para construir un ser vivo, ya sea una bacteria, un girasol, tú o un brontosaurio.

Ahora, imagina que encuentras un viejo manual, pero la mitad de las páginas están dañadas o faltan. Eso es lo que pasa con el ADN de dinosaurios. Aunque hemos encontrado ámbar con mosquitos que chuparon la sangre de dinosaurios (¡sí, como en la película!), el ADN en él está superdañado. Y, para empeorar las cosas, el ADN no dura eternamente. Después de un tiempo, se rompe en pedacitos, y después de millones de años, bueno... es más difícil de encontrar que un calcetín.

Entonces, aunque tengamos fragmentos de ADN de dinosaurio, reconstruir todo el manual es como intentar armar el mueble con solo un tercio de las instrucciones y sin saber qué mueble es.

Pero, supongamos que, mágicamente, logramos tener todo el ADN. Ahí no terminan los problemas. Para que un dinosaurio nazca, necesitas mucho más que solo su ADN. Es como tener el manual, pero sin las herramientas, tornillos o incluso el espacio adecuado para armarlo. Necesitaríamos un óvulo de dinosaurio (que, obviamente, no tenemos) y una máquina del tiempo para traer a una madre dinosaurio dispuesta a incubar el huevo.

Ahora, hablemos de la parte de "crear organismos en el laboratorio". En teoría, con el ADN correcto y las herramientas adecuadas, podríamos hacerlo. De hecho, ya estamos experimentando con cosas como ovejas clonadas y edición genética. Pero de ahí a recrear un organismo extinto hay un gran trecho. Y, honestamente, aunque pudiéramos... ¿Deberíamos? Un mundo con dinosaurios suena genial en el cine, pero en la vida real, habría un montón de problemas éticos y prácticos. ¿Dónde vivirían? ¿Cómo los alimentaríamos? Y, lo más importante, ¿cómo nos aseguramos de que no se coman a los turistas?

En resumen, aunque la idea de un Jurassic Park es emocionante, la ciencia detrás de ello es complicada y, por ahora, está más en el terreno de la ciencia ficción que en la realidad. Así que, por el momento, lo más cercano que tendremos a un dinosaurio será ese pollo rostizado del domingo. Después de todo, ¡los pájaros son descendientes directos de los dinosaurios! Y, si piensas en ello, eso es igual de impresionante.
