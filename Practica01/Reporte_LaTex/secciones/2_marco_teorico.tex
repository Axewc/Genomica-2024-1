%%%%%%%%%%%%%%%%%%%%%%%%%%%%%%%%%%%%%%%%%%%%%%%%%
\section*{Marco teórico} % NOMBRE DE LA SECCIÓN
\label{sec:marco_teorico} % ETIQUETA
%%%%%%%%%%%%%%%%%%%%%%%%%%%%%%%%%%%%%%%%%%%%%%%%%

Un \textit{gen} es una unidad de herencia que ocupa una ubicación específica (locus) en un cromosoma. En el contexto de la biología molecular, un gen es una secuencia de nucleótidos en el ADN que codifica la síntesis de una cadena de polipéptidos o de una molécula de ácido ribonucleico (ARN) con una función conocida.

Un \textbf{Marco de Lectura Abierto} (ORF, por sus siglas en inglés, \textit{Open Reading Frame}) es una secuencia continua de nucleótidos que tiene la potencialidad de codificar una proteína. Un ORF comienza con un codón de inicio (generalmente ATG, que codifica para el aminoácido metionina) y termina con uno de los tres codones de parada (TAA, TAG o TGA), sin ningún otro codón de parada en medio. Un ORF, por lo tanto, representa una parte de la secuencia de un gen que tiene el potencial de codificar una proteína. Para lograr este trabajo, se siguieron los siguientes pasos:
